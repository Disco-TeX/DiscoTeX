\documentclass[11pt]{article}

\begin{document}
\section{Introducing \DiscoTeX}
Mathematicians and other technical scientists often write their papers in \TeX{} and \LaTeX.
The most common output format for \TeX{}-based documents is PDF, which can be shared easily via email, and printed out into a physical copy.
While PDF excels as a format for physical documents, the experience of reading a technical paper on a computer can be significantly improved.
Those who read on a tablet or computer are aware of the fact that PDFs lack interactivity, such as text reflow, navigation history, and easy reference viewing.

By contrast, modern web technologies allow us to display content custom to our needs.
\DiscoTeX{} is an attempt to compile \TeX{} into well-designed webpages.
There are two parts to this project:
\begin{enumerate}
    \item Compiling a \TeX{} file into a human-readable HTML document, and
    \item Using web-based tools, to turn this HTML document into a well-designed interactive webpage which is pleasurable to read.
\end{enumerate}
This demonstration is intended to showcase both of these functionalities.
This entire HTML page was generated by the \DiscoTeX{} compilation engine, from \href{demo.tex}{this \TeX{} source file}.
If you view the source code, you'll see that the auto-generated HTML makes heavy use of custom HTML tags in order to make it easy to read.
We use \href{https://angularjs.org/}{AngularJS} to turn this HTML into a pretty and interactive webpage.

\section{Three ways to write beatiful webpages}
Because \DiscoTeX{} provides a readable and writable intermediate HTML format, there are now three ways to write your documents.
\begin{enumerate}
    \item Write \TeX{} and compile it with \DiscoTeX{}. Save the resulting intermediate HTML as your webpage.
    \item Write the intermediate HTML format directly.
    \item Write \TeX{} and follow the instructions \href{??}{here} to have your website dynamically load, compile, and render your document.
\end{enumerate}


\end{document}
